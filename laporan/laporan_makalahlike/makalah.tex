\documentclass[a4paper]{article}

% page margin
\usepackage[a4paper]{geometry}
\usepackage[utf8]{inputenc}
\usepackage{graphicx} % image importing stuff
\usepackage{mathptmx} % use adobe times roman
\usepackage{listings} % for code blocks
\usepackage{indentfirst} % indent first line on paraghraph
\usepackage{sectsty} % for modifying section
\usepackage{setspace} % buat spacing
\usepackage{array} % buat tabel array

\allsectionsfont{\centering}
\usepackage[
backend=biber,
style=authoryear-ibid,
]{biblatex}
\addbibresource{refs.bib}
\onehalfspacing
\def\arraystretch{1.5}
\renewcommand*\contentsname{Daftar Isi}

\begin{document}

\begin{center}

%------------------------ Cover -----------------------

  %------------------------ Judul -----------------------

  {\huge \bfseries Laporan Acara 9 dan 10 PHP}
  \\[0.7em]

  %------------------------ Mata Kuliah -----------------------

  Disusun untuk memenuhi tugas mata kuliah {\it Workshop Sistem Informasi Berbasis Web}
  \\[0.3em]

  %------------------------ Nama Dosen -----------------------

  Dosen Pengampu: {Khafidurrohman Agustianto, S.Pd, M.Eng.}
  \\[5em]

  %------------------------ Logo Poliyeay -----------------------

  \includegraphics[width=13em]{figs/poliyeay-bw.png}
  \\[3em]

  %------------------------ Pribadi -----------------------

  {\large Raihan Achmad Dani Firmansyah}
  \\[0.7em]
  {\large E41210369}
  \\[13em]

  %------------------------ Kampus/Prodi/Jurusan -----------------------

  {\Large Program Studi Teknik Informatika}
    \\[0.7em]
  {\Large Jurusan Teknik Informasi}
    \\[0.7em]
  {\Large Politeknik Negeri Jember}
    \\[0.7em]
  {\Large 2021}

\end{center}

%------------------------ Daftar Isi -----------------------

\newpage
\begin{center}
\tableofcontents
\end{center}

%------------------------ Pendahuluan -----------------------

\newpage
\section{Pendahuluan}
PHP: Hypertext Preprocessor (sebelumnya disebut Personal Home Pages) atau hanya PHP saja, adalah bahasa skrip dengan fungsi umum yang terutama digunakan untuk pengembangan web. Bahasa ini awalnya dibuat oleh seorang pemrogram Denmark-Kanada Rasmus Lerdorf pada tahun 1994. Implementasi referensi PHP sekarang diproduksi oleh The PHP Group. PHP awalnya merupakan singkatan dari Personal Home Page, tetapi sekarang merupakan singkatan dari inisialisasi rekursif PHP: Hypertext Preprocessor. \autocite{wikiphp}

% %------------------------ Hasil Praktik -----------------------

\newpage
\section{Hasil Praktik}

\hfill \break

\subsection{Acara 9}
  \begin{center}

  \begin{enumerate}
    \item 

  \end{enumerate}

  \end{center}
\subsection{Acara 10}
  \begin{center}

  \begin{enumerate}
    \item 


  \end{enumerate}

  \end{center}

%------------------------ Kesimpulan -----------------------

  \newpage
  \section{Kesimpulan}
    OOP lebih cepat dan lebih mudah untuk dieksekusi, OOP menyediakan struktur yang jelas untuk program OOP membantu menjaga kode PHP DRY "Don't Repeat Yourself", dan membuat kode lebih mudah untuk di-develop, dimodifikasi, dan di-debug. OOP memungkinkan untuk membuat aplikasi penuh yang dapat digunakan kembali dengan lebih sedikit kode dan waktu pengembangan yang lebih singkat \cite{w3php}


%------------------------ Daftar Pustaka -----------------------

  \newpage

  % ==== stack nocite disini aja
  \nocite{bkpmphp}
  \nocite{Austin2019}
  % \nocite{wstack}

  \begin{center}
  \printbibliography[heading=bibintoc,title={Daftar Pustaka}]
  \end{center}

%------------------------ End of Document -----------------------


%------------------------ Components -----------------------

% ==== tabel sama itemize ====
%     \begin{tabular}
% {| m{3cm} | m{6cm} | m{6cm} | }
%       \hline
%        &  &  \\
%       \hline
%        & 
%       \begin{itemize} 
%           \item 
%           \item 
%       \end{itemize} & 
%       \begin{itemize} 
%           \item 
%       \end{itemize} \\
%       \hline
%   \end{tabular}

%-----------------------------------------------------------

\end{document}