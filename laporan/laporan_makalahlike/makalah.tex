\documentclass[a4paper]{article}

\usepackage[a4paper]{geometry}
\usepackage{graphicx}
\usepackage{mathptmx}
\usepackage{listings}
\usepackage{indentfirst}
\usepackage{sectsty}
\allsectionsfont{\centering}
\usepackage[backend=biber
]{biblatex}
\addbibresource{refs.bib}

\renewcommand*\contentsname{Daftar Isi}

\begin{document}

\begin{center}

%------------------------ Cover -----------------------

  %------------------------ Judul -----------------------
  {\huge \bfseries Judul Laporan}
  \\[0.7em]
  %------------------------ Mata Kuliah -----------------------
  Disusun untuk memenuhi tugas mata kuliah {\it Judul Mata Kuliah}
  \\[0.3em]
  %------------------------ Nama Dosen -----------------------
  Dosen Pengampu: Nama Dosen
  \\[10em]

  %------------------------ Logo Poliyeay -----------------------
  \includegraphics[width=15em]{figs/poliyeay.png}
  \\[3em]
  %------------------------ Pribadi -----------------------
  {\large Raihan Achmad Dani Firmansyah}
  \\[0.7em]
  {\large E41210369}
  \\[15em]

  %------------------------ Kampus/Prodi/Jurusan -----------------------
  {\Large Program Studi Teknik Informatika}
    \\[0.7em]
  {\Large Jurusan Teknik Informasi}
    \\[0.7em]
  {\Large Politeknik Negeri Jember}
    \\[0.7em]
  {\Large 2021}

\end{center}

%------------------------ Daftar Isi -----------------------
\newpage
\begin{center}
\tableofcontents
\end{center}

%------------------------ Pendahuluan -----------------------
\newpage
\section{Pendahuluan}

%------------------------ Hasil Praktik -----------------------
\newpage
\section{Hasil Praktik}

\begin{enumerate}
  \item 
  {\bfseries Soal:} Tanda “ ” pada nilai variabel biasanya menandakan sebuah string, tetapi pada saat
  dilakukan operasi penjumlahan ternyata menghasilkan nilai jumlah seperti pada
  umumnya penjumlahan aritmatika, mengapa bisa terjadi? (variable dan tipe data)
  \\
  {\bfseries Jawaban:} Karena jika variable itu adalah bertipe data String dan dilakukan operasi dengan operator aritmatika penjumlahan maka variabel tersebut akan dikonversikan ke tipe data Integer

  \begin{lstlisting}[breaklines=true]
    <?php 
      $satu = 90;
      $dua = 80;

      echo $satu . $dua;
    ?>
  \end{lstlisting}

\end{enumerate}
%------------------------ Kesimpulan -----------------------
  \newpage
  \section{Kesimpulan}

%------------------------ Daftar Pustaka -----------------------
  \newpage

  \nocite{bkpm}
  \nocite{wstack}

  \begin{center}
  \printbibliography[heading=bibintoc,title={Daftar Pustaka}]
  \end{center}

\end{document}